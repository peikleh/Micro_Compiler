
\documentclass[11pt, oneside]{article}   	% use "amsart" instead of "article" for AMSLaTeX format
\usepackage{geometry}                		% See geometry.pdf to learn the layout options. There are lots.
\geometry{letterpaper}                   		% ... or a4paper or a5paper or ... 
%\geometry{landscape}                		% Activate for rotated page geometry
%\usepackage[parfill]{parskip}    		% Activate to begin paragraphs with an empty line rather than an indent
\usepackage{graphicx}				% Use pdf, png, jpg, or eps§ with pdflatex; use eps in DVI mode
								% TeX will automatically convert eps --> pdf in pdflatex		
\usepackage{amssymb}

%SetFonts

%SetFonts


\title{Crafting a Compiler}
\author{Nevin Leh\\Aaron Newhall\\Tim Weghen}
%\date{}							% Activate to display a given date or no date

\begin{document}
\maketitle
\section{Introduction}
The purpose of this project is to create a compiler. This is essentially a translation of a human readable language into a computer readable language. This is being done to explore the realm of computing at the most basic level and to illustrate the process of creating a compiler from start to finish.
\section{Background}
\section{Methods and Discussion}
\subsection{Scanner}
Creating a scanner is the first step of creating a compiler. The scanner can be compared to the dictionary in ordinary language. It simple finds all the legal words and symbols(called tokens) that make up a program and rejects a program that contains invalid tokens.

	For this compiler the lex program from the PLY package provide by python was used. This tool was chosen for it's simplicity and ease of use in creating a scanner. It is very similar to the lex program provided with most unix systems and has a relatively easy setup compared to other scanner generators such as ANTLR. Initially ANTLR was the chosen scanner for this project but the very specific dependencies proved to be too complicated and a switch to PLY was decided.
	
Implementing the scanner was pretty simple. The provided guide was used heavily to get used to using PLY and to visualize what was needed to use PLY Every token was defined such as keyword or operator and a regular expression was used to define each one of these rules. Most of the regular expressions were pretty simple with a couple exceptions. One has to be careful that a token is not matched by a different token. For example, "ENDWHILE" will be matched by the rule that defines "END". To get around this a lookahead was used so "END" would only match if it wasn't followed by "WHILE". Another tricky part was finding a way of excluding the comments from the output file. It was finally determined that a prefix of t\_error was needed for the comment regular expression so it would be ignored.
	
	
\subsection{Parser}
\subsection{Symbol Table}
\subsection{Semantic Routines}
\subsection{Full-fledged Compiler}

\section{Conclusion and Future Work}

%\subsection{}



\end{document}  